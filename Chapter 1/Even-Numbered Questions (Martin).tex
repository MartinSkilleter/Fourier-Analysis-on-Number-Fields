\documentclass[11pt,oneside]{article}

\usepackage{amssymb, amsmath, amsfonts, amsthm, latexsym, enumitem}
\usepackage[left=3cm, right=3cm, top=3cm, bottom=3cm]{geometry}
\usepackage{ mathrsfs }
%\usepackage{tikz-cd}
%\usepackage[alphabetic]{amsrefs}
%\usepackage{calrsfs}
%\usepackage{graphicx}
\usepackage{color}

\def\R{{\mathbb R}}
\def\Q{{\mathbb Q}}
\def\Z{{\mathbb Z}}
\def\C{{\mathbb C}}
\def\S{{\mathbb S}}
\def\N{{\mathbb N}}
\def\H{{\mathbb H}}
\def\F{{\mathbb F}}
\def\PP{\mathbb P}
\def\A{{\mathbb A}}
\def\p{\partial}
\def\O{\Omega}
\def\a{\alpha}
\def\b{\beta}
\def\g{\gamma}
\def\G{\Gamma}
\def\d{\delta}
\def\D{\Delta}
\def\e{\varepsilon}
\def\r{\rho}
\def\t{\tau}
\def\l{\lambda}
\def\L{\Lambda}
\def\k{\kappa}
\def\s{\sigma}
\def\th{\theta}
\def\o{\omega}
\def\z{\zeta}
\def\n{\nabla}
\def\T{{\mathcal T}}
\def\X{{\mathcal X}}
\def\area{\mathrm{area}}
\def\dist{\mathrm{\rm dist}}
\def\diag{\mathrm{diag}}
\def\spt{\mathrm{spt\,}}
\def\diam{\mathrm{diam\,}}
\def\dim{\mathrm{dim\,}}
\def\graph{\mathrm{graph\,}}
\def\interior{\mathrm{interior\,}}
\def\diag{\mathrm{diag\,}}
\def\Image{\mathrm{Image}\,}
\def\osc{\mathop{\text{\rm osc}}}
\def\id{\operatorname{id}}
\def\im{\operatorname{im}}
\def\ker{\operatorname{ker}}
\def\rank{\operatorname{rank}}
\def\dim{\operatorname{dim}}
\def\Re{\operatorname{Re}}
\def\Im{\operatorname{Im}}
\def\Aut{\operatorname{Aut}}
\def\Hom{\operatorname{Hom}}
\def\End{\operatorname{End}}
\def\GL{\operatorname{GL}}
\def\SL{\operatorname{SL}}

\def\suff{\implies}
\def\notsuff{\centernot\suff}
\def\nec{\impliedby}
\def\notnec{\centernot\nec}

\def\ra{\rightarrow}
\def\rai{\rightarrow\infty}

\newenvironment{solution}{\par\color{blue}\textbf{Solution:}}{\qed\color{black}}
\newenvironment{Part}[1]{\par\color{blue}\textbf{Part #1: }}{\qed}
\newenvironment{lemma}[1]{\par\color{blue}\textbf{Lemma#1: }}{}
\newenvironment{lemmaproof}{\par\color{blue}\textbf{Proof: }}{\qed}
\newenvironment{corollary}{\par\color{blue}\textbf{Corollary: }}

\usepackage{sectsty}
\allsectionsfont{\normalfont\scshape}

\overfullrule0pt 
\hoffset=0cm
\parskip=4pt
\parindent=0pt
\baselineskip=20pt

\begin{document}

2. Let $G = \Z$ and impose the following topology: $U \subseteq G$ is open if either $0 \notin U$ or $G - U$ is finite. Show that $G$ is \textit{not} a topological group with respect to this topology.

\begin{proof}
Suppose for sake of contradiction that $G$ is a topological group, and let $\tau : G \to G$ denote the map $a \mapsto a+1$, which is a homeomorphism of $G$. 

Consider the set $U = \{-1\} \subseteq G$, which is open because $0 \notin U$. But $\tau(U) = \{0\}$ is not open because $0 \in \tau(U)$, and $G - \tau(U) = \Z -\{0\}$ is not finite. Since $\tau$ is a homeomorphism and hence an open map, this is a contradiction.
\end{proof}

4. Give an example of a topological group with a closed subgroup that is \textit{not} open.

\begin{proof}
Consider $G = \GL_n(\C)$. Then $H = \SL_n(\C)$ is a closed subgroup since it is the kernel of $\det : \GL_n(\C) \to \C$. However, it cannot be open because $G$ is connected, so cannot have a proper clopen subset.

\end{proof}

6. Let $G = \GL_n(\R)$. Show that $G^0$ is the set of $n \times n$ matrices with positive determinant.

\begin{proof}
Let $G^+$ and $G^-$ denote the $n \times n$ invertible matrices with positive and negative determinant respectively. Observe that $G^+$ is homeomorphic to $G^-$ via the map which is multiplication by
\[ \left( \begin{matrix} 
-1 & 0 & ... & 0 \\
0 & 1 & ... & 0 \\
\vdots & 0 & \ddots & 0 \\
0 & 0 & ... & 1
\end{matrix} \right). \]
Hence it suffices to show that $G^+$ is connected, so that $G^+$ and $G^-$ are the connected components of $G$. Clearly $e \in G^+$, so we then have $G^0 = G^+$.
\end{proof}

\end{document}
